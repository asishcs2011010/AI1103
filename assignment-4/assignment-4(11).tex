\documentclass[journal,12pt,twocolumn]{IEEEtran}

\usepackage{setspace}
\usepackage{gensymb}
\singlespacing
\usepackage[cmex10]{amsmath}

\usepackage{amsthm}
\usepackage{mathrsfs}
\usepackage{txfonts}
\usepackage{stfloats}
\usepackage{bm}
\usepackage{cite}
\usepackage{cases}
\usepackage{subfig}
\usepackage{graphicx}
\usepackage{longtable}
\usepackage{multirow}

\usepackage{enumitem}
\usepackage{mathtools}
\usepackage{steinmetz}
\usepackage{tikz}
\usepackage{circuitikz}
\usepackage{verbatim}
\usepackage{tfrupee}
\usepackage[breaklinks=true]{hyperref}
\usepackage{graphicx}
\usepackage{tkz-euclide}

\usetikzlibrary{cal]yuc,math}
\usepackage{listings}
    \usepackage{color}                                            %%
    \usepackage{array}                                            %%
    \usepackage{longtable}                                        %%
    \usepackage{calc}                                             %%
    \usepackage{multirow}                                         %%
    \usepackage{hhline}                                           %%
    \usepackage{ifthen}                                           %%
    \usepackage{lscape}     
\usepackage{multicol}
\usepackage{chngcntr}
\newtheorem{theorem}{Theorem}[section]
\newtheorem{lemma}[theorem]{Lemma}
\DeclareMathOperator*{\Res}{Res}

\renewcommand\thesection{\arabic{section}}
\renewcommand\thesubsection{\thesection.\arabic{subsection}}
\renewcommand\thesubsubsection{\thesubsection.\arabic{subsubsection}}

\renewcommand\thesectiondis{\arabic{section}}
\renewcommand\thesubsectiondis{\thesectiondis.\arabic{subsection}}
\renewcommand\thesubsubsectiondis{\thesubsectiondis.\arabic{subsubsection}}


\hyphenation{op-tical net-works semi-conduc-tor}
\def\inputGnumericTable{}                                 %%

\lstset{
%language=C,
frame=single, 
breaklines=true,
columns=fullflexible
}
\begin{document}

\newcommand{\BEQA}{\begin{eqnarray}}
\newcommand{\EEQA}{\end{eqnarray}}
\newcommand{\define}{\stackrel{\triangle}{=}}
\bibliographystyle{IEEEtran}
\raggedbottom
\setlength{\parindent}{0pt}
\providecommand{\mbf}{\mathbf}
\providecommand{\pr}[1]{\ensuremath{\Pr\left(#1\right)}}
\providecommand{\qfunc}[1]{\ensuremath{Q\left(#1\right)}}
\providecommand{\sbrak}[1]{\ensuremath{{}\left[#1\right]}}
\providecommand{\lsbrak}[1]{\ensuremath{{}\left[#1\right.}}
\providecommand{\rsbrak}[1]{\ensuremath{{}\left.#1\right]}}
\providecommand{\brak}[1]{\ensuremath{\left(#1\right)}}
\providecommand{\lbrak}[1]{\ensuremath{\left(#1\right.}}
\providecommand{\rbrak}[1]{\ensuremath{\left.#1\right)}}
\providecommand{\cbrak}[1]{\ensuremath{\left\{#1\right\}}}
\providecommand{\lcbrak}[1]{\ensuremath{\left\{#1\right.}}
\providecommand{\rcbrak}[1]{\ensuremath{\left.#1\right\}}}
\theoremstyle{remark}
\newtheorem{rem}{Remark}
\newcommand{\sgn}{\mathop{\mathrm{sgn}}}
\providecommand{\abs}[1]{\vert#1\vert}
\providecommand{\res}[1]{\Res\displaylimits_{#1}} 
\providecommand{\norm}[1]{\lVert#1\rVert}
%\providecommand{\norm}[1]{\lVert#1\rVert}
\providecommand{\mtx}[1]{\mathbf{#1}}
\providecommand{\mean}[1]{E[ #1 ]}
\providecommand{\fourier}{\overset{\mathcal{F}}{ \rightleftharpoons}}
%\providecommand{\hilbert}{\overset{\mathcal{H}}{ \rightleftharpoons}}
\providecommand{\system}{\overset{\mathcal{H}}{ \longleftrightarrow}}
	%\newcommand{\solution}[2]{\textbf{Solution:}{#1}}
\newcommand{\solution}{\noindent \textbf{Solution: }}
\newcommand{\cosec}{\,\text{cosec}\,}
\providecommand{\dec}[2]{\ensuremath{\overset{#1}{\underset{#2}{\gtrless}}}}
\newcommand{\myvec}[1]{\ensuremath{\begin{pmatrix}#1\end{pmatrix}}}
\newcommand{\mydet}[1]{\ensuremath{\begin{vmatrix}#1\end{vmatrix}}}
\numberwithin{equation}{subsection}
\makeatletter
\@addtoreset{figure}{problem}
\makeatother
\let\StandardTheFigure\thefigure
\let\vec\mathbf
\renewcommand{\thefigure}{\theproblem}
\def\putbox#1#2#3{\makebox[0in][l]{\makebox[#1][l]{}\raisebox{\baselineskip}[0in][0in]{\raisebox{#2}[0in][0in]{#3}}}}
     \def\rightbox#1{\makebox[0in][r]{#1}}
     \def\centbox#1{\makebox[0in]{#1}}
     \def\topbox#1{\raisebox{-\baselineskip}[0in][0in]{#1}}
     \def\midbox#1{\raisebox{-0.5\baselineskip}[0in][0in]{#1}}
\vspace{3cm}
\title{AI1103-Assignment 4}
\author{Name: Asish sashank reddy, Roll Number: CS20BTECH11010}
\maketitle
\newpage
\bigskip
\renewcommand{\thefigure}{\theenumi}
\renewcommand{\thetable}{\theenumi}
Download all python codes from 
\begin{lstlisting}
https://github.com/asishcs2011010/demo/blob/main/assignment-4/codes
\end{lstlisting}
%
and latex-tikz codes from 
%
\begin{lstlisting}
https://github.com/asishcs2011010/demo/blob/main/assignment-4/assignment-4(11).tex 
\end{lstlisting}
\section*{question no}
gov/stats/2015/statistics-I(1), Q.1(C)
\section*{Question}
1)(c) Let X have pdf
  \begin{center}
f(x)= 
\begin{cases}
   \frac{1}{3} & -1 \le x < 2 \\
   0 &  otherwise
\end{cases}
\end {center}
Obtain the cdf of Y=X$^2$
\section*{Solution}
\begin{lemma}
 The cdf of X is defined as,
\begin{align}
    \mbox F_X\brak{x}=\pr{X \le x}\\
   \pr{X \le x} = \int_{-\infty}^{x} \mbox f(x)dx\\
 \mbox F_X\brak{x} = \int_{-\infty}^{x} \frac{1}{3}dx
 \end{align}
 \end{lemma}
  \begin{center}
  $ \mbox F_X\brak{x}$=  
 \begin{cases}
   0 & x<-1 \\
  \int_{-1}^{x} \frac{1}{3}dx  & -1\le x < 2\\
  \int_{-1}^{2} \frac{1}{3}dx  &   x \ge 2
\end{cases}
\end{center}
  \begin{center}
 $\mbox  F_X\brak{x}$ =  
 \begin{cases}
  0 & x<-1 \\
  \frac{x+1}{3} & -1\le x < 2\\
   1  &   x \ge 2
\end{cases}
\end{center}
\begin{lemma}
The cdf of Y = $X^2$ is given by  $ \mbox F_Y\brak{y}$\\
\begin{align}
 \mbox F_Y\brak{y}= \mbox Pr(X^2 \le y) & =\mbox Pr(-\sqrt{y} \le X \le\sqrt{y})\\
  & =  \mbox F_X\brak{\sqrt y} - \mbox F_X\brak{-\sqrt y}
 \end{align}
 \end{lemma}
 
 \begin{center}
  $\mbox F_Y\brak{y}$ =  
 \begin{cases}
  0 & y<0 \\
  \left(  \frac{\sqrt{y}+1}{3} \right) - \left( \frac{-\sqrt{y}+1}{3} \right) & 0 \le y <1 \\
\left(  \frac{\sqrt{y}+1}{3} \right) - 0 & 1 \le y < 4\\
   1-0  &   y \ge 4
\end{cases}
\end{center}
The cdf of Y is 
\begin{center}
  $\mbox F_Y\brak{y}$ =  
 \begin{cases}
  0 & y<0 \\
  \frac{2\sqrt{y}}{3} & 0 \le y <1 \\
  \frac{\sqrt{y}+1}{3} & 1 \le y < 4\\
   1  &   y \ge 4
\end{cases}
\end{center}
The plot of CDF of  X is given in the Figure \ref{fig:cdf}
\begin{figure}[h!]
\centering
\includegraphics[width=\columnwidth]{fig_X.png}
\caption{CDF of X}
\label{fig:cdf}
\end{figure}\\\\\\\\\\\\\\\\\\\\\\\\\\\\\\
The plot of CDF of Y is given in the Figure \ref{fig:cdf}
\begin{figure}[h!]
\centering
\includegraphics[width=\columnwidth]{fig_Y.png}
\caption{CDF of Y}
\label{fig:cdf}
\end{figure}
\end{document}
© 2021 GitHub, Inc.
Terms
Privacy
Security
Status
Docs
Contact GitHub
Pricing
API
Training
Blog
About