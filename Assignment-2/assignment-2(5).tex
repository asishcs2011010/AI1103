\documentclass[journal,12pt,twocolumn]{IEEEtran}

\usepackage{setspace}
\usepackage{gensymb}
\singlespacing
\usepackage[cmex10]{amsmath}

\usepackage{amsthm}

\usepackage{mathrsfs}
\usepackage{txfonts}
\usepackage{stfloats}
\usepackage{bm}
\usepackage{cite}
\usepackage{cases}
\usepackage{subfig}

\usepackage{longtable}
\usepackage{multirow}

\usepackage{enumitem}
\usepackage{mathtools}
\usepackage{steinmetz}
\usepackage{tikz}
\usepackage{circuitikz}
\usepackage{verbatim}
\usepackage{tfrupee}
\usepackage[breaklinks=true]{hyperref}
\usepackage{graphicx}
\usepackage{tkz-euclide}

\usetikzlibrary{calc,math}
\usepackage{listings}
    \usepackage{color}                                            %%
    \usepackage{array}                                            %%
    \usepackage{longtable}                                        %%
    \usepackage{calc}                                             %%
    \usepackage{multirow}                                         %%
    \usepackage{hhline}                                           %%
    \usepackage{ifthen}                                           %%
    \usepackage{lscape}     
\usepackage{multicol}
\usepackage{chngcntr}

\DeclareMathOperator*{\Res}{Res}

\renewcommand\thesection{\arabic{section}}
\renewcommand\thesubsection{\thesection.\arabic{subsection}}
\renewcommand\thesubsubsection{\thesubsection.\arabic{subsubsection}}

\renewcommand\thesectiondis{\arabic{section}}
\renewcommand\thesubsectiondis{\thesectiondis.\arabic{subsection}}
\renewcommand\thesubsubsectiondis{\thesubsectiondis.\arabic{subsubsection}}


\hyphenation{op-tical net-works semi-conduc-tor}
\def\inputGnumericTable{}                                 %%

\lstset{
%language=C,
frame=single, 
breaklines=true,
columns=fullflexible
}
\begin{document}

\newcommand{\BEQA}{\begin{eqnarray}}
\newcommand{\EEQA}{\end{eqnarray}}
\newcommand{\define}{\stackrel{\triangle}{=}}
\bibliographystyle{IEEEtran}
\raggedbottom
\setlength{\parindent}{0pt}
\providecommand{\mbf}{\mathbf}
\providecommand{\pr}[1]{\ensuremath{\Pr\left(#1\right)}}
\providecommand{\qfunc}[1]{\ensuremath{Q\left(#1\right)}}
\providecommand{\sbrak}[1]{\ensuremath{{}\left[#1\right]}}
\providecommand{\lsbrak}[1]{\ensuremath{{}\left[#1\right.}}
\providecommand{\rsbrak}[1]{\ensuremath{{}\left.#1\right]}}
\providecommand{\brak}[1]{\ensuremath{\left(#1\right)}}
\providecommand{\lbrak}[1]{\ensuremath{\left(#1\right.}}
\providecommand{\rbrak}[1]{\ensuremath{\left.#1\right)}}
\providecommand{\cbrak}[1]{\ensuremath{\left\{#1\right\}}}
\providecommand{\lcbrak}[1]{\ensuremath{\left\{#1\right.}}
\providecommand{\rcbrak}[1]{\ensuremath{\left.#1\right\}}}
\theoremstyle{remark}
\newtheorem{rem}{Remark}
\newcommand{\sgn}{\mathop{\mathrm{sgn}}}
\providecommand{\abs}[1]{\vert#1\vert}
\providecommand{\res}[1]{\Res\displaylimits_{#1}} 
\providecommand{\norm}[1]{\lVert#1\rVert}
%\providecommand{\norm}[1]{\lVert#1\rVert}
\providecommand{\mtx}[1]{\mathbf{#1}}
\providecommand{\mean}[1]{E[ #1 ]}
\providecommand{\fourier}{\overset{\mathcal{F}}{ \rightleftharpoons}}
%\providecommand{\hilbert}{\overset{\mathcal{H}}{ \rightleftharpoons}}
\providecommand{\system}{\overset{\mathcal{H}}{ \longleftrightarrow}}
	%\newcommand{\solution}[2]{\textbf{Solution:}{#1}}
\newcommand{\solution}{\noindent \textbf{Solution: }}
\newcommand{\cosec}{\,\text{cosec}\,}
\providecommand{\dec}[2]{\ensuremath{\overset{#1}{\underset{#2}{\gtrless}}}}
\newcommand{\myvec}[1]{\ensuremath{\begin{pmatrix}#1\end{pmatrix}}}
\newcommand{\mydet}[1]{\ensuremath{\begin{vmatrix}#1\end{vmatrix}}}
\numberwithin{equation}{subsection}
\makeatletter
\@addtoreset{figure}{problem}
\makeatother
\let\StandardTheFigure\thefigure
\let\vec\mathbf
\renewcommand{\thefigure}{\theproblem}
\def\putbox#1#2#3{\makebox[0in][l]{\makebox[#1][l]{}\raisebox{\baselineskip}[0in][0in]{\raisebox{#2}[0in][0in]{#3}}}}
     \def\rightbox#1{\makebox[0in][r]{#1}}
     \def\centbox#1{\makebox[0in]{#1}}
     \def\topbox#1{\raisebox{-\baselineskip}[0in][0in]{#1}}
     \def\midbox#1{\raisebox{-0.5\baselineskip}[0in][0in]{#1}}
\vspace{3cm}
\title{AI1103-Assignment 2}
\author{Name: Asish sashank reddy, Roll Number: CS20BTECH11010}
\maketitle
\newpage
\bigskip
\renewcommand{\thefigure}{\theenumi}
\renewcommand{\thetable}{\theenumi}
Download latex-tikz codes from 
%
\begin{lstlisting}
https://github.com/asishcs2011010/demo/blob/main/Assignment-2/assignment-2(5).tex
\end{lstlisting}
\section*{question no}
Gate-EC Q-38
\section*{Question}
Let  $X\in \{ 0,1 \}$ and $Y\in \{ 0,1 \}$ be two independent binary random variables. if P$(X=0)$ = p and  P$(Y=0)$ = q, then P($X+Y \geqslant 1$) is equal to\\\\
\begin{enumerate}
\item $pq+(1-p)(1-q)$ \\
\item  $pq$    \\
\item $p(1-q)$ \\
\item  $1-pq$  \\
\end{enumerate}
\section*{Solution}
Given X,Y$\in \{0,1\}$ be two independent random variables. The probability mass function (pmf) is expressed  as 
\begin{align}
    p_{X}(n) = \pr{X = n} = 
\begin{cases}
p & n=0
\\
1-p & n=1
\end{cases}\label{1}
\end{align}
\begin{align}
    p_{Y}(n) = \pr{Y = n} = 
\begin{cases}
q & n=0
\\
1-q & n=1
\end{cases}\label{1}
\end{align}
The Z-transform of $p_X(n)$ is defined as
\begin{align}
P_X(z) = \sum_{n = -\infty}^{\infty}p_X(n)z^{-n}, \quad z \in \mathbb{C}
\label{7}
\end{align}
\begin{align}
P_X(z) = p+(1-p)z
\end{align}
similarly,
\begin{align}
P_Y(z) = q+(1-q)z    
\end{align}
Let Z be the convolution of X,Y.
\begin{align}
  Z=X+Y,
  \end{align}
  The probability mass function of Z is \\
  \begin{align}
  \mbox Pr(Z = z) = \sum_{k}\mbox Pr(X = K)\times \mbox Pr(Y = z-k) \label{1}
  \end{align}
  equation  \eqref{1} can be written as following using convolution operation
    \begin{align}
     \mbox P_Z(z) &= \mbox P_{X}(z)\times \mbox P_{Y}(z)  
    \end{align}
    \begin{align}
    \mbox P_Z(z)=(p+(1-p)z)\times(q+(1-q)z)\\
    \mbox P_Z(z)= pq+(p+q-2pq)z+(1-p)(1-q)z^2     
    \end{align}
    The pmf of Z is 
\begin{align}
    p_{Z}(n) = \pr{Z = n} = 
\begin{cases}
pq & n=0
\\
p+q-2pq & n=1\\
(1-p)(1-q) & n=2
\end{cases}\label{2}
\end{align}
\begin{align}
  \mbox Pr(Z<1) = \mbox P(Z=0)  \label{3} 
  \end{align}
  From equation \eqref{3}, we get
\begin{align}
  \mbox Pr(X+Y\geq1) = 1 - \mbox Pr(X+Y<1)\\
  \mbox Pr(Z\geq1)= 1-\mbox Pr(Z<1) = 1-pq   
\end{align}



\end{document}
© 2021 GitHub, Inc.
Terms
Privacy
Security
Status
Docs
Contact GitHub
Pricing
API
Training
Blog
About
